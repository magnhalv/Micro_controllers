\subsection{Implementation}
The implementation and some design issues of the required modules for 
this exercise is 
described within the following subsections.

\subsubsection{Branch and jump}
One of the challenges with pipelining is branch and jump instructions, since 
it interrupts the flow of instructions in the pipeline.

In our implementation of branching we predict branch taken, since this will 
be the best optimization for for- and while-loops. Thus a branch instruction 
will cause the PC register to branch on the decode stage, and if the 
prediction 
was wrong it will branch back to the correct instruction. This is done by 
inspecting the branch signal and the ALU\_op.Zero on the memory stage. If the 
prediction is wrong; we update the PC, set the control signals to the two 
former instructions to low, and signal the if\_id\_reg to no-op the next 
instruction.

Jumps are implemented by performing the jump in the decode stage, and simply 
bubbling the next instruction that comes from the instruction memory.  

\subsubsection{Fetch and Decode Register}
The point of the \textit{if\_id\_reg} is to provide support for no-ops and stalling 
instructions. No-ops are performed when the CPU is branching back after
a wrongly predicted branch. Stalling is performed when an instruction
needs to wait for a load. 

The compiler provides two warnings when synthesizing this component, 
because the \textit{proc\_enable}, \textit{no\_op} and \textit{if\_id\_write}
to the  sensitivity list. This is done on purpose, because we do not
want the current instruction in the decode stage to be updated when 
these signals are set.

\subsubsection{Decode and Execute Register}
The \textit{id\_ex\_reg} transfers all the signals from the decode stage to the 
execute stage. It also provides support for bubbling instructions due to 
branching back, a load dependency or jump. 

\subsubsection{Execute and Memory Register}
The \textit{ex\_mem\_reg} transfers all the signals from the execute stage, and 
provides support for bubbling instructions due to branch back. 

\subsubsection{Program Counter Register}
The program counter registers task is to increment the PC address, perform 
jumps, branches and branch backs. For jump and branch it receives all the 
signals it needs from the decode stage (control signals and addresses) to 
calculate the next instruction address. The signals it needs to decide 
whether to branch back or not it will receive from the memory stage. If a 
branch back is performed, it will signal the \textit{if\_id\_reg}, the 
\textit{id\_ex\_reg} and 
the \textit{ex\_mem\_reg} to perform no ops. 

\subsubsection{Control Unit}
This unit works more or less the same as in the last assignment.
The only exception is that it now keeps its own state, and that it
does not signal the PC register whether it should write or 
not.

\subsubsection{Hazard Detection Unit}
This unit detects when there is a load dependency, which makes it signal the 
\textit{if\_id\_reg}
not to update the current instruction on the next cycle. Then it stalls the 
PC register and bubbles the current instruction in decode. The unit also
 signals the 
\textit{ex\_mem\_reg} to bubble the current instruction when a jump has been 
performed.

\subsubsection{Forwarding unit}
This unit forwards data to the execution stage if one of the parameters of the ALU 
depends on the two previous instructions. It can forward the write back data 
from both the ALU result in the memory stage or the write back data in the 
write back stage. 

\subsubsection{Register File}
We edited the register file to write back on falling edge. Thus making it 
possible to update a register, and propagate the new value to the execution 
stage in one clock cycle. This reduces the architecture complexity, since we 
do not need to stall instructions due to write back dependencies.
