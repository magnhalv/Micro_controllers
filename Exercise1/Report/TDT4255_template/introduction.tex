\section{Introduction}

The components used in this assignement is the \textit{EFM32GG Giant Gecko microcontroller} combined with
the \textit{ARM Cortex-M3 processor}, connected to a \textit{gamepad} containing eight LEDs and buttons. The programming
is done in assembly and can be uploaded to the board by usb and a program called \textit{eACommander}. The processor supports the ARM Thumb-2 instruction set, so all assembly code must be written in it. 

The microcontroller is connected to the gamepad with a ribbon cable split into two parts, one for the buttons and one for the LEDs. The parts are connected to two different I/O ports on the microcontroller, port C and A, respectively. The buttons are connected to pin 0-7 on port C and the LEDs are connected to pin 8-15 on port A. 

Both the micocontroller and the processor are designed to be highly energy efficient, and thus support five different energy modes. Energy mode 0 provides full functionality, while energy mode 4 provides little functionality but uses a lot less power. Thus, in order to save power, we wish to keep the microcontroller in as high energy mode as possible at all times. 

We were provided with a framework which contained two assembly files, one containing useful constants and one containing enough assembly code to set up the microcontroller. The framework also contained a makefile which could be used to assemble, link and upload the code to the microcontroller. We were also provided with a compendium\cite{compendium}, containing useful information about the 
system, and reference guides\cite{cortex}\cite{efm32gg}.

\begin{figure}[ht]
    \centering
      \includegraphics[width=9cm]{figures/placeholder}
    \caption{The EF32GG microcontroller, connected to the gamepad.}
    \label{figure_ef32gg}
\end{figure}

\subsection{Assignment}
The assignment is as follows:

\begin{itemize}
    \item   Write an assembly program which enables a user to control the LEDs in some way by pressing the buttons.
    \item   Write an interrupt rotuine (an interrupt handler) for reading the buttons.
    \item   Use a Makefile to compile and upload the program to the microcontroller.
    \item   Use the energy monitor to see the power requirements of your program. Analyze and
            discuss (or implement) improvements. 
    \end{itemize}


The solution provided in this report have completed all of the listed conditions, and have also added some extra functionality. In addition to simply using the buttons to control the LEDs, they are also manipulated by interrupts from a countdown timer. This extra functionality will be described in more detail in the following sections. 