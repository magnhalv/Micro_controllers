	\section*{Abstract}

As the world today becomes more and more digitialized and depended on computers for efficiency,
embededded systems have become  increasingly popular. One of the main reasons for this is the
microcontroller, a small computer on a single integrated curcuit. Since microcontrollers are flexible 
and programmable, you can use the same microcontroller in several different embedded systems. Thus reducing
project and product cost, since you do not have to spend time designing a special purpose processor. 

But due to the vast increase of computer usage and the use of computers in extreme environments, power consumption 
and longtivity has become a major challenge within embedded computer design. 


This report covers the first exercise in the course TDT4258 Energy Efficient Computer Systems at NTNU. The
main purpose of this exercise is to get an introduction and general understanding of the architecture of \textit{the 
ARM Cortex-M3 microcontroller} and \textit{the EFM32GG DK3750 development board}. Thus this report will guide you through our 
process of making a simple program that uses the microcontroller to turn on LEDs when a button is pressed. We will also
present the design choices made in order to use as little power as possible. 


% Modern processors are powerful, yet relatively simple machines. What makes them so 
% powerful, is the possibility to execute millions of instructions per second.
% This is achieved through several performance optimizations. One of those
% optimization techniques is called \textit{pipelining}. 
% Pipelining is an
% essential part of any processor that aims to have a good performance.

% This report covers the second exercise in the course TDT4255 at NTNU,
% and builds upon the work done in the first exercise. The task in the first
% exercise was to create a MIPS multi cycle processor, and getting it to run
% on an FPGA. The task covered in this report was to add and implement a
% pipeline to the CPU created in the first exercise.

% This report covers all the different components and registers that has to 
% be implmented in order to extend a MIPS multi cycle processor with a 
% pipeline. It also concludes that an added pipeline has a significant increase 
% in performance, compared to a regular multi cycle processor.