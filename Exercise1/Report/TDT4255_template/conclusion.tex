\section{Conclusion}
We always strive to make processors as efficient as possible, and an
important step in this process is to add a pipeline to the processor.

We have implemented all the components and logic required by the specification 
of the assignment. Everything worked during the simulation in ModelSim.
The pipeline made for quite a lot of synchronization issues that had to be
considered and made for a lot of debugging, but satisfying results in the
end.

We saw that the pipelined processors performance greatly out matched the 
multi cycle processor from the first exercise. Thus we can conclude that this 
implementation have been a success. It is very motivating to see that all the
work put into the exercises makes for such satisfying results. 

It has been a lot of hard work, that has provided us with a lot of insight
in the complexity of designing a pipelined processor. Designing such 
a processor is a very complex task that requires a lot of work, testing and 
a great understanding in how the different components \textit{actually} 
work together. 

The course and its exercises has also given us a greater understanding of how a 
computer work at a much lower level than we have previously thought of,
when working on other computer-related
problems like e.g. software development for embedded devices, or even higher
level system programming. Even if we do not \textit{have} to take the processor
architecture into account when designing a software system, it is certain
that an understanding of \textit{how}
it works at lower abstraction levels can help when 
optimizing arbitrary problems.

All taken into considerations, we can say that the exercises in this
course has given us a great insight in processors and computer architecture,
VHDL coding and design of logical circuits. 